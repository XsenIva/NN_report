\documentclass[bachelor, och, referat]{shiza}
% параметр - тип обучения - одно из значений:
%    spec     - специальность
%    bachelor - бакалавриат (по умолчанию)
%    master   - магистратура
% параметр - форма обучения - одно из значений:
%    och   - очное (по умолчанию)
%    zaoch - заочное
% параметр - тип работы - одно из значений:
%    referat    - реферат
%    coursework - курсовая работа (по умолчанию)
%    diploma    - дипломная работа
%    pract      - отчет по практике
% параметр - включение шрифта
%    times    - включение шрифта Times New Roman (если установлен)
%               по умолчанию выключен
\usepackage{subfigure}
\usepackage{tikz,pgfplots}
\pgfplotsset{compat=1.5}
\usepackage{float}

%\usepackage{titlesec}
\setcounter{secnumdepth}{4}
%\titleformat{\paragraph}
%{\normalfont\normalsize}{\theparagraph}{1em}{}
%\titlespacing*{\paragraph}
%{35.5pt}{3.25ex plus 1ex minus .2ex}{1.5ex plus .2ex}

\titleformat{\paragraph}[block]
{\hspace{1.25cm}\normalfont}
{\theparagraph}{1ex}{}
\titlespacing{\paragraph}
{0cm}{2ex plus 1ex minus .2ex}{.4ex plus.2ex}

% --------------------------------------------------------------------------%


\usepackage[T2A]{fontenc}
\usepackage[utf8]{inputenc}
\usepackage{graphicx}
\graphicspath{ {./images/} }
\usepackage{tempora}

\usepackage[sort,compress]{cite}
\usepackage{amsmath}
\usepackage{amssymb}
\usepackage{amsthm}
\usepackage{fancyvrb}
\usepackage{listings}
\usepackage{listingsutf8}
\usepackage{longtable}
\usepackage{array}
\usepackage[english,russian]{babel}

%\usepackage[colorlinks=true]{hyperref}
\usepackage{url}

\usepackage{underscore}
\usepackage{setspace}
\usepackage{indentfirst} 
\usepackage{mathtools}
\usepackage{amsfonts}
\usepackage{enumitem}
\usepackage{tikz}
\newcommand{\eqdef}{\stackrel {\rm def}{=}}
\newcommand{\specialcell}[2][c]{%
\begin{tabular}[#1]{@{}c@{}}#2\end{tabular}}

\renewcommand\theFancyVerbLine{\small\arabic{FancyVerbLine}}

\newtheorem{lem}{Лемма}

\begin{document}

% Кафедра (в родительном падеже)
\chair{теоретических основ компьютерной безопасности и криптографии}

% Тема работы
\title{Коллаборативная фильтрация в рекомендательных системах}

% Курс
\course{5}

% Группа
\group{531}

% Факультет (в родительном падеже) (по умолчанию "факультета КНиИТ")
\department{факультета КНиИТ}

% Специальность/направление код - наименование
%\napravlenie{09.03.04 "--- Программная инженерия}
%\napravlenie{010500 "--- Математическое обеспечение и администрирование информационных систем}
%\napravlenie{230100 "--- Информатика и вычислительная техника}
%\napravlenie{231000 "--- Программная инженерия}
\napravlenie{100501 "--- Компьютерная безопасность}

% Для студентки. Для работы студента следующая команда не нужна.
\studenttitle{cтудентки}

% Фамилия, имя, отчество в родительном падеже
\author{Ивановой Ксении Всладиславовны}

%Научный руководитель (для реферата преподаватель проверяющий работу)
\satitle{Доцент} %должность, степень, звание
\saname{И. И. Слеповичев}

% Руководитель практики от организации (только для практики,
% для остальных типов работ не используется)
%\patitle{к.ф.-м.н.}
%\paname{М.~Б.~Абросимов}

% Семестр (только для практики, для остальных
% типов работ не используется)
%\term{8}

% Наименование практики (только для практики, для остальных
% типов работ не используется)
%\practtype{преддипломная}

% Продолжительность практики (количество недель) (только для практики,
% для остальных типов работ не используется)
%\duration{4}

% Даты начала и окончания практики (только для практики, для остальных
% типов работ не используется)
%\practStart{30.04.2019}
%\practFinish{27.05.2019}

% Год выполнения отчета
\date{2024}

\maketitle

% Включение нумерации рисунков, формул и таблиц по разделам
% (по умолчанию - нумерация сквозная)
% (допускается оба вида нумерации)
% \secNumbering

%-------------------------------------------------------------------------------------------
\tableofcontents

\intro

Одна из интересных тем современного мира, которая стоит на стыке информационных технологий и маркетинга,
не дающая уснуть ночью, подкидывая нам интересные видео, и помогающая тратить наши деньги, предлагая товары,
которые нам \glqq могут понравиться \grqq, тема рекомендательных систем. 

Сейчас, в период информационной зависимости, вопрос 
\glqq что надеть\grqq, смещается более актуальными: \glqq что посмотреть\grqq, а продвижение товара преобразуется из 
телевизионно-зомбирующей рекламы совершенно ненужных вам вещей в предложения купить товары с оценкой 5 звезд 
из 5. Умные алгоритмы рекомендаций товаров на основе каких-то действий или признаков покупателя помогают как бизнесу, 
так и потребителям. Одним из таких методов является коллаборативная фильтрация, о которой и пойдет речь.


\section{Метод коллаборативная фильтрации}
К коллаборативной (совместной) фильтрации относятся те методы и
алгоритмы, которые основываются на данных о
предыдущих сеансах работы пользователей с этой же
системой.

Общей чертой всех методов коллаборативной фильтрации является то, что основой создания
рекомендаций является пересечение оценочных мер популярности того или иного предмета. Если вам понравилось что-то один раз, 
то с большой вероятностью вам можно рекомендовать это снова и снова. Что обуславливается нейрофизиологией нашего мозга, 
который предпочитает энергоэффективную и стабильную привычность, а не что-то неизведанное.


Происходит построение некоторой матрицы предпочтений (user-item matrix) 
для предметов пользователями. Затем он сопоставляет пользователей с 
соответствующими интересами и предпочтениями, вычисляя сходства между их профилями
 для составления рекомендаций . Такие пользователи строят группу под названием 
 окрестности. Пользователь получает рекомендации к тем элементам, которые он не оценил 
 раньше, но которые уже были положительно оценены пользователями в его районе. 
 Рекомендации, которые вырабатываются CF, могут быть либо предсказанием, либо 
 рекомендацией. Предсказание является числовое значениеrij, выражая предсказанную 
 оценку элемента j для пользователя i, в то время как Рекомендация представляет 
 собой список лучших N элементов, которые пользователю понравятся больше всего, 
 как показано в Рис. 3. Технику коллаборативной фильтрации можно разделить на две
  категории: на основе памяти и на основе моделей 

\subsection{Корреляционные модели}
Корреляционные модели
(Memory-Based Collaborative Filtering)
хранение всей исходной матрицы данных F;
сходство клиентов — это корреляция строк матрицы F;
сходство объектов — это корреляция столбцов матрицы F.

\subsection{Латентные модели}
(Latent Models for Collaborative Filtering)
оценивание профилей клиентов и объектов
(профиль — это вектор скрытых характеристик);
хранение профилей вместо хранения F;
сходство клиентов и объектов — это сходство их профилей.

\subsection{Метода альтернативных наименьших квадратов}


\section{Алгоритмы}

методы факторизации на основе пользователей, 
методы факторизации на основе элементов, 
метод матричной факторизации.


\begin{thebibliography}{3}
  \bibitem{1}  
  Смоленчук Татьяна Владимировна Метод коллаборативной фильтрации для рекомендательных сервисов // Вестник науки и образования. 2019. №22-1 (76). URL: https://cyberleninka.ru/article/n/metod-kollaborativnoy-filtratsii-dlya-rekomendatelnyh-servisov (дата обращения: 18.12.2024).
  \bibitem{2}
  Ларионов В. С., Дунин И. В. ОБЗОР МЕТОДОВ КОЛЛАБОРАТИВНОЙ ФИЛЬТРАЦИИ // Форум молодых ученых. 2017. №5 (9). URL: https://cyberleninka.ru/article/n/obzor-metodov-kollaborativnoy-filtratsii (дата обращения: 18.12.2024). 
  \bibitem{3}
  Гомзин А. Г., Коршунов А. В. Системы рекомендаций: обзор современных подходов // Труды ИСП РАН. 2012. №. URL: https://cyberleninka.ru/article/n/sistemy-rekomendatsiy-obzor-sovremennyh-podhodov (дата обращения: 18.12.2024).
  \bibitem{4}
 https://cts.etu.ru/assets/files/2021/cts21/papers/287-290.pdf
  % \bibitem{1}
    % Катрина Уэйкфилд "Гид: алгоритмы машинного обучения и их типы" [Статья] – URL: https://www.sas.com/ru_ru/insights/articles/analytics/machine-learning-algorithms-guide.html (дата обращения 10.01.2024) - Загл. с экрана. Яз. рус.
    % \bibitem{2}
    % Документация PyTorch [Электронный ресурс] – URL: https://pytorch.org/ (дата обращения 12.01.2024) Яз. англ.
    % \bibitem{3}
    % OpenAI Five [Электронный ресурс] – URL: https://openai.com/five/ (дата обращения 12.01.2024) - Загл. с экрана. Яз. англ.
    % \bibitem{4}
    % D. Biswas "Self-improving Chatbots based on Deep Reinforcement Learning" [Статья] – URL: https://towardsdatascience.com/self-improving-chatbots-based-on-reinforcement-learning-75cca62debce (дата обращения 13.01.2024) - Загл. с экрана. Яз. англ.
    % \bibitem{5}
    % M. Berk "How to Use Reinforcement Learning to Recommend Content" [Статья] – URL: https://towardsdatascience.com/how-to-use-reinforcement-learning-to-recommend-content-6d7f9171b956 (дата обращения 13.01.2024) - Загл. с экрана. Яз. англ.
    % \bibitem{6}
    % Что не так с обучением с подкреплением (Reinforcement Learning)? [Электронный ресурс] – URL: https://habr.com/ru/post/437020/ (дата обращения 13.01.2024) - Загл. с экрана. Яз. рус.
    % \bibitem{7}
    % Intro to Policy Optimization [Электронный ресурс] – URL: https://spinningup.openai.com/en/latest/spinningup/rl_intro3.html (дата обращения 13.01.2024) - Загл. с экрана. Яз. англ.
    % \bibitem{8}
    % Evaluation metrics for reinforcement algorithms [Электронный ресурс] – URL: https://medium.com/@barathchandarcse/evaluation-metrics-for-reinforcement-algorithms-ff2bf5869fe4 (дата обращения 13.01.2024) - Загл. с экрана. Яз. англ.
\end{thebibliography}


\end{document}